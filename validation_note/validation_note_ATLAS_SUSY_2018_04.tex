%%%%%%%%%%%%%%%%%%%%%%%%%%%%%%%%%%%%%%%%%%%%%%%%%%%%%%%%%%%%%%%%%%%%%%%%%%%%
%% Trim Size: 9.75in x 6.5in
%% Text Area: 8in (include Runningheads) x 5in
%% ws-mpla.tex   :   29-9-2008
%% TeX file to use with ws-mpla.cls written in Latex2E.
%% The content, structure, format and layout of this style file is the
%% property of World Scientific Publishing Co. Pte. Ltd.
%% Copyright 1995, 2002 by World Scientific Publishing Co.
%% All rights are reserved.
%%%%%%%%%%%%%%%%%%%%%%%%%%%%%%%%%%%%%%%%%%%%%%%%%%%%%%%%%%%%%%%%%%%%%%%%%%%%
%%

\documentclass{ws-mpla}
\usepackage[super]{cite}
\usepackage{graphicx}
\begin{document}

\markboth{Jongwon Lim, Chih-Ting Lu, Jae-hyeon Park, and Jiwon Park}{IMPLEMENTATION OF THE ATLAS-SUSY-2018-04 ANALYSIS IN THE MADANALYSIS 5 FRAMEWORK}

%%%%%%%%%%%%%%%%%%%%% Publisher's Area please ignore %%%%%%%%%%%%%%
\catchline{}{}{}{}{}
%%%%%%%%%%%%%%%%%%%%%%%%%%%%%%%%%%%%%%%%%%%%%%%%%%%%%%%%%%%%%%%%%%%

\title{IMPLEMENTATION OF THE ATLAS-SUSY-2018-04 ANALYSIS IN THE MADANALYSIS 5 FRAMEWORK}

\author{\footnotesize Jongwon Lim}
\address{
  Department of Physics, Hanyang University, Seoul 04763, Republic of Korea}

\author{\footnotesize Chih-Ting Lu}
\address{
  School of Physics, KIAS, Seoul 02455, Republic of Korea}

\author{\footnotesize Jae-hyeon Park}
\address{
  School of Physics, KIAS, Seoul 02455, Republic of Korea}

\author{\footnotesize Jiwon Park}
\address{
  Department of Physics, Hanyang University, Seoul 04763, Republic of Korea}
\maketitle

\pub{Received (Day Month Year)}{Revised (Day Month Year)}

\begin{abstract}
We present the MADANALYSIS 5 implementation and validation of the ATLAS-SUSY-2018-04 search.
This ATLAS analysis targets the search for direct stau production in events with two hadronic tau leptons and probes 139 fb$^{-1}$ of LHC proton-proton collisions at a center-of-mass energy of 13 TeV.
The validation of our reimplementation relies on a comparison of our cutflow predictions with the auxiliary table of official ATLAS results in the context of two supersymmetry-inspired simplified benchmark models in which the Standard Model is extended by a neutralino and a stau decaying into a tau lepton and a neutralino.
\keywords{supersymmetry; stau; hadronic tau lepton.}
\end{abstract}

%\ccode{PACS Nos.: include PACS Nos.}

\section{Introduction}

\begin{figure}[t]
  \centerline{\includegraphics[width=2.0in]{fig_01}}
  \vspace*{8pt}
  \caption{The Feynman diagram for the process $pp\rightarrow\tilde{\tau}\tilde{\tau}\rightarrow\tilde{\chi}^0_1\tilde{\chi}^0_1\tau\tau$.\protect\label{fig:fig_01}}
\end{figure}

In this note, we describe the validation of the implementation, in MADANALYSIS 5 framework~\cite{Conte:2018vmg,Dumont:2014tja,Conte:2014zja,Conte:2012fm}, of the ATLAS-SUSY-2018-04 analysis~\cite{Aad:2019byo} with direct stau production in two hadronic $\tau +E^{miss}_T$ events. 
This process is illustrated by the representative Feynman diagram of Fig.~\ref{fig:fig_01}. 
This analysis focuses on LHC proton-proton collisions at a center-of-mass energy of 13 TeV and an integrated luminosity of $139 fb^{-1}$.

For the validation of our reimplementation, we have focused on the sector of sparticles with only electroweak interactions. 
The lightest neutralino ($\tilde{\chi}^0_1$) is taken as the lightest supersymmetric particle (LSP). 
The stau-left ($\tilde{\tau}_L$) and stau-right ($\tilde{\tau}_R$) are combined with the assumption of mass degenerate and no mixing is assumed between the gauge eigenstates ($\tilde{\tau}_L$,$\tilde{\tau}_R$) and mass eigenstates ($\tilde{\tau}_1$,$\tilde{\tau}_2$). 
Furthermore, in order to suppress other decay modes of stau, the masses of all charginos and neutralinos are set to 2.5 TeV except for the $\tilde{\chi}^0_1$. 
Hence, the single kinematically allowed decay mode of stau is 
\begin{equation}
\tilde{\tau}\rightarrow\tilde{\chi}^0_1 \tau 
\end{equation}
 

\section{Description of the analysis}

This analysis targets a final state containing two hadronic tau leptons with a certain amount of missing transverse energy. 
The kinematics of di-$\tau +E^{miss}_T$ system is used to reduce the contributions from Standard Model backgrounds. 
First, all the objects are reconstructed and defined. Then a sequence of event selections for the signal final state is applied.

\subsection{Object definitions}

Jets are reconstructed by means of the anti-$k_t$ algorithm~\cite{Cacciari:2008gp} with a radius parameter set to $R=0.4$. This analysis focuses on jets whose transverse momentum $p^j_T$ and pseudorapidity $\eta^j$ fullfill
\begin{equation}
p^j_T > 20 \textrm{GeV}\quad \textrm{and}\quad |\eta^j| < 2.8.
\end{equation} 
Moreover, the selected jets are tagged as originating from the fragmentation of a $b$-quark with 
\begin{equation}
p^b_T > 20 \textrm{GeV}\quad \textrm{and}\quad |\eta^b| < 2.5.
\end{equation}
A working point with the average $b$-tagging efficiency of $77\%$ is used. This working point corresponds to a $c$-jet and light-jet rejection of $4.9$ and $110$, respectively.

Electron candidates are required to have a transverse momentum $p^e_T$ and pseudorapidity $\eta^e$ obeying
\begin{equation}
p^e_T > 17 \textrm{GeV}\quad \textrm{and}\quad |\eta^e| < 2.47.
\end{equation}
Furthermore, all electron candidates are required to have both track and calorimeter isolations. The condition of the track isolation is
\begin{equation}
\sum p_{T,\textrm{tracks}}/p^e_T < 0.15\quad \textrm{with}\quad \Delta R=min(10\textrm{GeV}/p^e_T,0.2),
\end{equation}
the condition of the calorimeter isolation is
\begin{equation}
\sum E_{T,\textrm{calorimeter}}/p^e_T < 0.2\quad \textrm{with}\quad \Delta R=0.2,
\end{equation}
and for high transverse momentum electron, 
\begin{equation}
\sum E_{T,\textrm{tracks}} < max(0.015\times p^e_T,3.5\textrm{GeV})\quad \textrm{with}\quad \Delta R=0.2\quad \textrm{if}\quad p^e_T > 200\textrm{GeV}.
\end{equation}

Muon candidate definition is similar, although with slightly looser thresholds,
\begin{equation}
p^{\mu}_T > 14 \textrm{GeV}\quad \textrm{and}\quad |\eta^{\mu}| < 2.7,
\end{equation}
The condition of the track isolation is 
\begin{equation}
\sum p_{T,\textrm{tracks}}/p^{\mu}_T < 0.15\quad \textrm{with}\quad \Delta R=min(10\textrm{GeV}/p^{\mu}_T,0.3),
\end{equation}
and the condition of the calorimeter isolation is
\begin{equation}
\sum E_{T,\textrm{tracks}}/p^{\mu}_T < 0.3\quad \textrm{with}\quad \Delta R=0.2.
\end{equation}

Tau lepton candidates are reconstructed with one or three associated charged pion tracks (prongs). %$\sum_i e_i (\textrm{tracks}) = \pm 1$.
%For 1-prong (3-prong) $\tau$ lepton candidates, the signal efficiencies are $75\%$($60\%$) and $60\%$($45\%$) for the \textit{medium} and \textit{tight} working points, respectively.
For 1-prong (3-prong) $\tau$ lepton candidates, the signal efficiencies are $75\%$ and $60\%$ for the \textit{medium} working points. Due to the technical restriction, the \textit{tight} working point efficiencies are directly taken from the official ATLAS cutflow table. %Should add more details
The baseline tau lepton candidates are required to have 
\begin{equation}
p^{\tau}_T > 50(40) \textrm{GeV}\quad \textrm{and}\quad |\eta^{\tau}| < 2.5
\end{equation}
for the leading (subleading) ones and the transition region between the barrel and endcap calorimeters ($ 1.37 < |\eta^{\tau}| < 1.52 $) is excluded.

Finally, some overlap removal conditions are in order. Tau lepton is removed if $\Delta R(\tau,e/\mu) < 0.2$, electron or muon is removed if $\Delta R(e/\mu,j) < 0.4$, and jet is removed if $\Delta R(\tau,j) < 0.4$.


\subsection{Event selection}
%The analysis contains two \textit{medium} tau lepton candidates with opposite-sign electric charge (OS). All events are required to pass either an \textit{asymmetric di-$\tau$} trigger for the low stau mass region (SR-lowMass) or a combined \textit{di-$\tau +E^{miss}_T$} ($E^{miss}_T > 150$ GeV) trigger for the high stau mass region (SR-highMass). The trigger efficiencies about $80\%$ are applied in our recasting with the following offline $p_T$ thresholds for the leading (subleading) tau lepton candidates in Table~\ref{tab:trig-eff}.
%Events with the third \textit{medium} tau lepton or any light lepton are rejected. 
%On the other hand, values of $75 < E^{miss}_T < 150$ GeV are required for SR-lowMass to increase signal sensitivity.

%Furthermore, $b$-jet veto is applied to reject events from top quark associated processes. The reconstructed invariant mass of the two leading tau lepton candidates, $m(\tau_1,\tau_2)$, larger than $120$ GeV is required for removing tau lepton pair from low-mass resonances, $Z$ boson, and Higgs boson events ($Z/H$ veto).

%Here mostly re-arranged sentences from above paragraphs
The events with exactly two \textit{medium} tau lepton candidates with opposite-sign electric charge (OS) are selected. 
Then, $b$-jet veto is applied to reject events from top quark associated processes.
Also, the events with additional light lepton (muon or electron) are rejected.
The reconstructed invariant mass of the two leading tau lepton candidates, $m(\tau_1,\tau_2)$, larger than $120$ GeV is required for removing tau lepton pair from low-mass resonances, $Z$ boson, and Higgs boson events ($Z/H$ veto).

All events are required to pass either an \textit{asymmetric di-$\tau$} trigger for the low stau mass region (SR-lowMass) or a combined \textit{di-$\tau +E^{miss}_T$} ($E^{miss}_T > 150$ GeV) trigger for the high stau mass region (SR-highMass) (\textbf{Trigger and offline cuts}).
The trigger efficiencies about $80\%$ are applied in our recasting with the following offline $p_T$ thresholds for the leading (subleading) tau lepton candidates in Table~\ref{tab:trig-eff}.

\begin{table}[h!]
  \tbl{Offline $p_T$ thresholds for the leading (subleading) tau lepton candidates of \textit{asymmetric di-$\tau$} and \textit{di-$\tau +E^{miss}_T$} triggers with efficiencies about $80\%$.}
  {\begin{tabular}{@{}c c c@{}} \toprule
  Year & \textit{asymmetric di-$\tau$} & \textit{di-$\tau +E^{miss}_T$} \\
  \colrule
 2015-2017 & $95(60)$ GeV & $50(40)$ GeV \\
 2018 & $95(75)$ GeV & $75(40)$ GeV \\ 
  \botrule
  \end{tabular}\label{tab:trig-eff} }
\end{table}

In SR-lowMass region, values of $75 < E^{miss}_T < 150$ GeV are required for SR-lowMass to increase signal sensitivity.
Also, two selected tau leptons are required to be tight tagged.
The selection efficiency of two taus passing the \textit{tight} working point on top of two medium tagged tau leptons is taken from official ATLAS cutflow table as a ratio of raw number of event before and after applying cut, and the tight tagging efficiency is then applied as a probability per tau lepton by a square root of the efficiency.
On the other hand, in SR-highMass region, the tight tagging efficiency is applied in the same manner as SR-lowMass region, but allowing at least one of two tau leptons passing the tight selection.

The \textit{stransverse mass} $m_{T2}$ variable is defined as\footnote{
Notice $m_{T2}$ calculation is done with MADANALYSIS 5 function ($PHYSICS\rightarrow Transverse\rightarrow MT2(vec1,vec2,E^{miss}_T,m_{invisible})$)
}
\begin{equation}
m_{T2} =min_{\mathbf{q}_T}
\left[
max(m_{T,\tau_1}(\mathbf{p}_{T,\tau_1},\mathbf{q}_T),m_{T,\tau_2}(\mathbf{p}_{T,\tau_2},\mathbf{p}^{miss}_T -\mathbf{q}_T))
\right],
\end{equation}   
where $\mathbf{p}_{T,\tau_1}$ and $\mathbf{p}_{T,\tau_2}$ are the transverse momenta of the two tau lepton candidates, and the transverse momentum vector of one of the invisible particle, $\mathbf{q}_T$, is choosen to minimize the larger of the two transverse mass $m_{T,\tau_1}$ and $m_{T,\tau_2}$. The transverse mass $m_T$ is defined by
\begin{equation}
m_{T}(\mathbf{p}_T,\mathbf{q}_T) = \sqrt{2(p_T q_T -\mathbf{p}_T\cdot\mathbf{q}_T)}.
\end{equation} 
A lower bound on the $m_{T2}$ will be imposed to reduce contributions from $t\overline{t}$ and $WW$ events.
Finally, the two tau lepton candidates are required to satisfy $\Delta R(\tau_1,\tau_2) < 3.2$, $|\Delta\phi (\tau_1,\tau_2)| > 0.8$ and $m_{T2} > 70$ GeV to further suppress contributions from SM backgrounds.


%\begin{table}[t]
%  \tbl{Please use this template for tables.}
%  {\begin{tabular}{@{}cccc@{}} \toprule
%  Piston mass & Analytical frequency & TRIA6-$S_1$ model &
%  \% Error \\
%  & (Rad/s) & (Rad/s) \\
%  \colrule
%  1.0\hphantom{00} & \hphantom{0}281.0 & \hphantom{0}280.81 & 0.07 \\
%  0.1\hphantom{00} & \hphantom{0}876.0 & \hphantom{0}875.74 & 0.03 \\
%  0.01\hphantom{0} & 2441.0 & 2441.0\hphantom{0} & 0.0\hphantom{0} \\
%  0.001 & 4130.0 & 4129.3\hphantom{0} & 0.16\\ \botrule
%  \end{tabular}\label{ta1} }
%\end{table}


\section{Validation}

\subsection{Event generation}

In order to validate our analysis, we rely on the MSSM UFO model file~\cite{Duhr:2011se} from Feynrules model database~\cite{Alloul:2013bka}. 
Two benchmark points with masses $ m(\tilde{\tau},\tilde{\chi}^0_1)=(120,1) $GeV and $ (280,1) $GeV are used in this note to illustrate the validation of our reimplementation. 
We make use of MADGRAPH5 aMC@NLO version 2.6.7~\cite{Alwall:2014hca} for hard-scattering event generation in which leading-order matrix elements are convoluted with the NNPDF23LO~\cite{Martin:2009iq} parton distribution function (PDF) set. The signal includes the emission of up to two additional partons. We apply the MLM scheme~\cite{Mangano:2006rw,Alwall:2008qv} of the ME-PS matching with $xqcut = m_{\tilde{\tau}}/4$. 
The PYTHIA8 version 8.244~\cite{Sjostrand:2007gs} with $A14$ tune has been used for the simulation of the parton showering and hadronization. The simulation of the detector response has been performed by using DELPHES-3.4.2~\cite{deFavereau:2013fsa}, that relies on FASTJET~\cite{Cacciari:2011ma} for object reconstruction.
The modified delphes card has been used with an appropriate tuned detector card.
For example, the loosened isolation criteria are applied to cover all offline object definitions.
And the radius parameter of jet and minimum transverse momentum are lowered to 0.4 and 15 GeV with updating $b$ and tau tagging efficiencies.
Also, UniqueObjectFinder is disabled for overlap removal which is done in MADANALYSIS5.
Finally, we have used the MADANALYSIS5 reimplementation to calculate the signal selection efficiencies.



\subsection{Comparison with the official results}
In Table~\ref{tab:120GeV} and~\ref{tab:280GeV}, we compare the results obtained with our implememtation to the official raw event numbers in the auxiliary tables provided by the ATLAS collaboration for the benchmark points with masses $m(\tilde{\tau},\tilde{\chi}^0_1)=(120,1) $ and $(280,1)$GeV, respectively. 
For each cut, we have calculated the related efficiency defined as 
\begin{equation}
\epsilon_i =\frac{n_i}{n_{i-1}}
\end{equation}
where $ n_i $ and $ n_{i-1} $ mean the event number after and before the considered cut, respectively.
%
On the other hand, we have also calculated the differences between $ \epsilon_i (MA5)$ and $ \epsilon_i (ATLAS)$ with the definition as
\begin{equation}
diff. = \frac{\epsilon_i (MA5)-\epsilon_i (ATLAS)}{\epsilon_i (ATLAS)}
\end{equation}


\begin{table}[h!]
  \tbl{Validation checks of the cut flows for $ \tilde{\tau}\tilde{\tau} $ production with $ m(\tilde{\tau},\tilde{\chi}^0_1) = (120,1) $ GeV.}
  {\begin{tabular}{@{}c c c c c c@{}} \toprule
\hline
\multicolumn{6}{c}{ \textbf{$ \tilde{\tau}\tilde{\tau} $ production with $ m(\tilde{\tau},\tilde{\chi}^0_1) = (120,1) $ GeV} }\\
\hline\hline
 & ATLAS($N_{raw}$) & $\epsilon_i$($\%$) & MA5($N_{raw}$) & $\epsilon_i$($\%$) & diff.($\%$) \\
\hline\hline

2 medium $\tau$ (OS) and 3rd baseline $\tau$ veto & 22493 & & 25857 & & \\ \hline
$b$-jet veto & 22148 & 98.47 & 25270 & 97.73 & $-0.75$ \\ \hline
Light lepton veto & 22109 & 99.82 & 25248 & 99.91 & 0.00 \\ \hline
$Z/H$-veto & 18188 & 82.27 & 20940 & 82.94 & 0.80 \\ \hline
%
\multicolumn{5}{c}{ \textbf{SR-lowMass} }\\\hline
%
Trigger and offline cuts & 6512 & 35.80 & 7675 & 36.65 & 2.30 \\ \hline
$ 75 < E^{miss}_T < 150 $ GeV & 2228 & 34.21 & 2653 & 34.57 & 1.00 \\ \hline
2 tight $\tau$ & 1565 & 70.24 & 1747 & 65.85 & $-6.25$ \\ \hline
$ |\Delta\phi(\tau,\tau)| > 0.8 $ & 1564 & 99.94 & 1744 & 99.83 & $-0.11$ \\ \hline
$ |\Delta R(\tau,\tau)| < 3.2 $ & 1429 & 91.37 & 1595 & 91.46 & 0.10 \\ \hline
$ m_{T2} > 70 $ GeV & 280 & 19.59 & 453 & 28.40 & 44.90 \\ \hline
All &  & 1.24 &  & 1.75 & 40.70 \\ \hline
%
\hline
\multicolumn{5}{c}{ \textbf{SR-highMass} }\\\hline
%
Trigger and offline cuts & 1272 & 6.99 & 1709 & 8.16 & 16.70 \\ \hline
$ \geq 1 $ tight $\tau$ & 1249 & 98.19 & 1632 & 95.49 & $-2.75$ \\ \hline
$ |\Delta\phi(\tau,\tau)| > 0.8 $ & 1236 & 98.96 & 1600 & 98.04 & $-0.93$ \\ \hline
$ |\Delta R(\tau,\tau)| < 3.2 $ & 1132 & 91.59 & 1444 & 90.25 & $-1.46$ \\ \hline
$ m_{T2} > 70 $ GeV & 170 & 15.02 & 345 & 23.89 & 59.00 \\ \hline
All &  & 0.76 &  & 1.33 & 76.50 \\ \botrule
\end{tabular}
\label{tab:120GeV} }
\end{table}

\begin{table}[h!]
  \tbl{Validation checks of the cut flows for $ \tilde{\tau}\tilde{\tau} $ production with $ m(\tilde{\tau},\tilde{\chi}^0_1) = (280,1) $ GeV.}
  {\begin{tabular}{@{}c c c c c c@{}} \toprule
\hline
\multicolumn{6}{c}{ \textbf{$ \tilde{\tau}\tilde{\tau} $ production with $ m(\tilde{\tau},\tilde{\chi}^0_1) = (120,1) $ GeV} }\\
\hline\hline
 & ATLAS($N_{raw}$) & $\epsilon_i$($\%$) & MA5($N_{raw}$) & $\epsilon_i$($\%$) & diff.($\%$) \\
\hline\hline

2 medium $\tau$ (OS) and 3rd baseline $\tau$ veto & 3980 & & 35034 & & \\ \hline
$b$-jet veto & 3903 & 98.07 & 34113 & 97.37 & $-0.71$ \\ \hline
Light lepton veto & 3888 & 99.62 & 34059 & 99.84 & 0.20 \\ \hline
$Z/H$-veto & 3382 & 86.99 & 30116 & 88.42 & 1.60 \\ \hline
%
\multicolumn{5}{c}{ \textbf{SR-lowMass} }\\\hline
%
Trigger and offline cuts & 1920 & 56.77 & 18677 & 62.02 & 9.20 \\ \hline
$ 75 < E^{miss}_T < 150 $ GeV & 738 & 38.44 & 6730 & 36.03 & $-6.25$ \\ \hline
2 tight $\tau$ & 512 & 69.38 & 4328 & 64.31 & $-7.30$ \\ \hline
$ |\Delta\phi(\tau,\tau)| > 0.8 $ & 512 & 100.00 & 4296 & 99.26 & $-0.74$ \\ \hline
$ |\Delta R(\tau,\tau)| < 3.2 $ & 478 & 93.36 & 3928 & 91.43 & $-2.06$ \\ \hline
$ m_{T2} > 70 $ GeV & 278 & 58.16 & 2441 & 62.14 & 6.80 \\ \hline
All &  & 6.98 &  & 6.97 & $-0.25$ \\ \hline
%
\hline
\multicolumn{5}{c}{ \textbf{SR-highMass} }\\\hline
%
Trigger and offline cuts & 1096 & 32.41 & 11526 & 38.27 & 18.10 \\ \hline
$ \geq 1 $ tight $\tau$ & 1076 & 98.18 & 11315 & 98.17 & $-0.01$ \\ \hline
$ |\Delta\phi(\tau,\tau)| > 0.8 $ & 1045 & 97.12 & 10833 & 95.74 & $-1.42$ \\ \hline
$ |\Delta R(\tau,\tau)| < 3.2 $ & 973 & 93.11 & 10140 & 93.60 & 0.50 \\ \hline
$ m_{T2} > 70 $ GeV & 691 & 71.02 & 7887 & 77.78 & 9.50 \\ \hline
All &  & 17.36 &  & 22.51 & 29.60 \\ \botrule
\end{tabular}
\label{tab:280GeV} }
\end{table} 


We observe that the disagreement on a cut-by-cut basis, is $-2.65\%$ and $10.03\%$ before the $m_{T2}$ cut for SR-lowMass and SR-highMass in Table~\ref{tab:120GeV} for $m(\tilde{\tau},\tilde{\chi}^0_1)=(120,1) $GeV. The major parts of the disagreement come from \textbf{Trigger and offline cuts}, and tight $\tau$ selection steps. By lack of more public experimental information, we have not been able to validate these two steps more precisely.
\begin{figure}[h!]
  \centerline{\includegraphics[width=2.0in]{m120_norm_1}\includegraphics[width=2.0in]{m120_norm_2}}
  \vspace*{8pt}
  \caption{The $m_{T2}$ distributions for $m(\tilde{\tau},\tilde{\chi}^0_1)=(120,1) $GeV.\protect\label{fig:m120_norm}}
\end{figure}

\begin{figure}[t]
  \centerline{\includegraphics[width=2.0in]{m280_norm_1}\includegraphics[width=2.0in]{m280_norm_2}}
  \vspace*{8pt}
  \caption{The $m_{T2}$ distributions for $m(\tilde{\tau},\tilde{\chi}^0_1)=(280,1) $GeV.\protect\label{fig:m280_norm}}
\end{figure}
In Fig.~\ref{fig:m120_norm}, we realize the $m_{T2}$ distributions from ATLAS analysis are sofer than our results. This causes the $m_{T2} > 70$GeV cut looser in our reimplementation than the original ATLAS results which is taken from HEPData\cite{hepdata}.
Similarly, the disagreement on a cut-by-cut basis is $-0.25\%$ and $29.60\%$ with all cuts for SR-lowMass and SR-highMass in Table~\ref{tab:280GeV} for $m(\tilde{\tau},\tilde{\chi}^0_1)=(280,1) $GeV. The major parts of the disagreement still come from  \textbf{Trigger and offline cuts}, tight $\tau$ selection, and $m_{T2}$ cut steps. 
The $m_{T2}$ distributions are shown in Fig.~\ref{fig:m280_norm} for the comparison of our results with ATLAS analysis. 

%\begin{figure}[t]
%  \centerline{\includegraphics[width=2.0in]{mplaf1}}
%  \vspace*{8pt}
%  \caption{Please use this template for figures.\protect\label{fig1}}
%\end{figure}

\section{Conclusions}

We have implemented the ATLAS-SUSY-2018-04 search in the MADANALYSIS 5 framework. Our analysis has been validated in the context of two supersymmetry-inspired simplified benchmark models in which the Standard Model is extended by a neutralino and a stau decaying into a tau lepton and a neutrino. 
By comparing our predictions for the cutflow with the official one provided by ATLAS in Ref.\cite{Aad:2019byo}, we have found an agreement for each step in Table~\ref{tab:120GeV} and~\ref{tab:280GeV} except for the ones from \textbf{Trigger and offline cuts}, tight $\tau$ selection, and $m_{T2}$ cut. Due to the lack of more information, we have not been able to validate these steps more precisely. 


\section*{Acknowledgments}
Dedications and funding information may be included here.

\begin{thebibliography}{99}
\bibitem{Conte:2018vmg}
  E.~Conte and B.~Fuks,
  Int.\ J.\ Mod.\ Phys.\ A {\bf 33} (2018) no.28,  1830027
  [arXiv:1808.00480 [hep-ph]].

\bibitem{Dumont:2014tja}
  B.~Dumont {\it et al.},
  Eur.\ Phys.\ J.\ C {\bf 75} (2015) no.2,  56
  [arXiv:1407.3278 [hep-ph]].

\bibitem{Conte:2014zja}
  E.~Conte, B.~Dumont, B.~Fuks and C.~Wymant,
  Eur.\ Phys.\ J.\ C {\bf 74} (2014) no.10,  3103
  [arXiv:1405.3982 [hep-ph]].

\bibitem{Conte:2012fm}
  E.~Conte, B.~Fuks and G.~Serret,
  Comput.\ Phys.\ Commun.\  {\bf 184} (2013) 222
  [arXiv:1206.1599 [hep-ph]].

%\cite{Aad:2019byo}
\bibitem{Aad:2019byo} 
  G.~Aad {\it et al.} [ATLAS Collaboration],
  %``Search for direct stau production in events with two hadronic $\tau$-leptons in $\sqrt{s} = 13$ TeV $pp$ collisions with the ATLAS detector,''
  Phys.\ Rev.\ D {\bf 101}, no. 3, 032009 (2020)
  doi:10.1103/PhysRevD.101.032009
  [arXiv:1911.06660 [hep-ex]].
  %%CITATION = doi:10.1103/PhysRevD.101.032009;%%
  %7 citations counted in INSPIRE as of 27 Mar 2020

%\cite{Aad:2019byo}
\bibitem{hepdata}
G.~Aad \textit{et al.} [ATLAS],
%``Search for direct stau production in events with two hadronic $\tau$-leptons in $\sqrt{s} = 13$ TeV $pp$ collisions with the ATLAS detector,''
doi:10.17182/hepdata.92006
%https://doi.org/10.17182/hepdata.92006

%\cite{Cacciari:2008gp}
\bibitem{Cacciari:2008gp} 
  M.~Cacciari, G.~P.~Salam and G.~Soyez,
  %``The anti-$k_t$ jet clustering algorithm,''
  JHEP {\bf 0804}, 063 (2008)
  doi:10.1088/1126-6708/2008/04/063
  [arXiv:0802.1189 [hep-ph]].
  %%CITATION = doi:10.1088/1126-6708/2008/04/063;%%
  %6790 citations counted in INSPIRE as of 27 Mar 2020

%\cite{Duhr:2011se}
\bibitem{Duhr:2011se} 
  C.~Duhr and B.~Fuks,
  %``A superspace module for the FeynRules package,''
  Comput.\ Phys.\ Commun.\  {\bf 182}, 2404 (2011)
  doi:10.1016/j.cpc.2011.06.009
  [arXiv:1102.4191 [hep-ph]].
  %%CITATION = doi:10.1016/j.cpc.2011.06.009;%%
  %69 citations counted in INSPIRE as of 27 Mar 2020
  
%\cite{Alloul:2013bka}
\bibitem{Alloul:2013bka} 
  A.~Alloul, N.~D.~Christensen, C.~Degrande, C.~Duhr and B.~Fuks,
  %``FeynRules  2.0 - A complete toolbox for tree-level phenomenology,''
  Comput.\ Phys.\ Commun.\  {\bf 185}, 2250 (2014)
  doi:10.1016/j.cpc.2014.04.012
  [arXiv:1310.1921 [hep-ph]].
  %%CITATION = doi:10.1016/j.cpc.2014.04.012;%%
  %1327 citations counted in INSPIRE as of 27 Mar 2020

%\cite{Alwall:2014hca}
\bibitem{Alwall:2014hca} 
  J.~Alwall {\it et al.},
  %``The automated computation of tree-level and next-to-leading order differential cross sections, and their matching to parton shower simulations,''
  JHEP {\bf 1407}, 079 (2014)
  doi:10.1007/JHEP07(2014)079
  [arXiv:1405.0301 [hep-ph]].
  %%CITATION = doi:10.1007/JHEP07(2014)079;%%
  %4502 citations counted in INSPIRE as of 27 Mar 2020
    
%\cite{Martin:2009iq}
\bibitem{Martin:2009iq} 
  A.~D.~Martin, W.~J.~Stirling, R.~S.~Thorne and G.~Watt,
  %``Parton distributions for the LHC,''
  Eur.\ Phys.\ J.\ C {\bf 63}, 189 (2009)
  doi:10.1140/epjc/s10052-009-1072-5
  [arXiv:0901.0002 [hep-ph]].
  %%CITATION = doi:10.1140/epjc/s10052-009-1072-5;%%
  %4803 citations counted in INSPIRE as of 27 Mar 2020  
  
%\cite{Mangano:2006rw}
\bibitem{Mangano:2006rw} 
  M.~L.~Mangano, M.~Moretti, F.~Piccinini and M.~Treccani,
  %``Matching matrix elements and shower evolution for top-quark production in hadronic collisions,''
  JHEP {\bf 0701}, 013 (2007)
  doi:10.1088/1126-6708/2007/01/013
  [hep-ph/0611129].
  %%CITATION = doi:10.1088/1126-6708/2007/01/013;%%
  %695 citations counted in INSPIRE as of 27 Mar 2020

%\cite{Alwall:2008qv}
\bibitem{Alwall:2008qv} 
  J.~Alwall, S.~de Visscher and F.~Maltoni,
  %``QCD radiation in the production of heavy colored particles at the LHC,''
  JHEP {\bf 0902}, 017 (2009)
  doi:10.1088/1126-6708/2009/02/017
  [arXiv:0810.5350 [hep-ph]].
  %%CITATION = doi:10.1088/1126-6708/2009/02/017;%%
  %210 citations counted in INSPIRE as of 27 Mar 2020  

%\cite{Sjostrand:2007gs}
\bibitem{Sjostrand:2007gs} 
  T.~Sjostrand, S.~Mrenna and P.~Z.~Skands,
  %``A Brief Introduction to PYTHIA 8.1,''
  Comput.\ Phys.\ Commun.\  {\bf 178}, 852 (2008)
  doi:10.1016/j.cpc.2008.01.036
  [arXiv:0710.3820 [hep-ph]].
  %%CITATION = doi:10.1016/j.cpc.2008.01.036;%%
  %5077 citations counted in INSPIRE as of 27 Mar 2020

%\cite{deFavereau:2013fsa}
\bibitem{deFavereau:2013fsa} 
  J.~de Favereau {\it et al.} [DELPHES 3 Collaboration],
  %``DELPHES 3, A modular framework for fast simulation of a generic collider experiment,''
  JHEP {\bf 1402}, 057 (2014)
  doi:10.1007/JHEP02(2014)057
  [arXiv:1307.6346 [hep-ex]].
  %%CITATION = doi:10.1007/JHEP02(2014)057;%%
  %1518 citations counted in INSPIRE as of 27 Mar 2020

%\cite{Cacciari:2011ma}
\bibitem{Cacciari:2011ma} 
  M.~Cacciari, G.~P.~Salam and G.~Soyez,
  %``FastJet User Manual,''
  Eur.\ Phys.\ J.\ C {\bf 72}, 1896 (2012)
  doi:10.1140/epjc/s10052-012-1896-2
  [arXiv:1111.6097 [hep-ph]].
  %%CITATION = doi:10.1140/epjc/s10052-012-1896-2;%%
  %3362 citations counted in INSPIRE as of 27 Mar 2020
\end{thebibliography}
\end{document}
